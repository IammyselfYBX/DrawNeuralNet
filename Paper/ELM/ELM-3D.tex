\documentclass[border=8pt, multi, tikz]{standalone}
\usepackage{import}
\usepackage{ctex}  % 引入ctex包以支持中文
\subimport{../../layers/}{init}
\usetikzlibrary{positioning, calc, 3d} % 引入3d库

% 定义颜色
\def\KernelColor{rgb:red,0;green,166;blue,82}
\def\ConvColor{rgb:yellow,5;red,2.5;white,5}
\def\ConvReluColor{rgb:yellow,5;red,5;white,5}
\def\PoolColor{rgb:red,1;black,0.3}
\def\UnpoolColor{rgb:blue,2;green,1;black,0.3}
\def\FcColor{rgb:blue,5;red,2.5;white,5}
\def\FcReluColor{rgb:blue,5;red,5;white,4}
\def\SoftmaxColor{rgb:magenta,5;black,7}

% 自定义箭头的命令
\newcommand{\copymidarrow}{\tikz \draw[-Stealth,line width =0.8mm,draw={rgb:blue,4;red,1;green,1;black,3}] (-0.3,0) -- ++(0.3,0);}


\begin{document}
\begin{tikzpicture}
\tikzstyle{connection}=[ultra thick,every node/.style={sloped,allow upside down},draw=\edgecolor,opacity=0.7]


% 总节点数
\def\numinputs{5}
\def\numhidden{6}
\def\numoutputs{2}

% 计算每一层的垂直中心,偶数和奇数判断
\pgfmathsetmacro{\inputcenter}{(\numinputs - 1) / 2}
\pgfmathsetmacro{\hiddencenter}{(\numhidden - 1) / 2}
\pgfmathsetmacro{\outputcenter}{(\numoutputs - 1) / 2}

% 输入层节点位置
\foreach \i in {1,...,\numinputs}
{
    \pgfmathsetmacro{\y}{\inputcenter - (\i - 1)} % 根据输入层的中点调整y坐标
    \pic[shift={(0,\y,5)}] at (0,\y,0) {Ball={name=input\i, fill=blue, opacity=0.6, radius=2.5, logo=$x_\i$}};
}
\node[font=\small, align=left] at (0, -5, 5) {输入层}; % 居中调整字体大小

% 隐藏层节点位置
\foreach \i in {1,...,\numhidden}
{
    \pgfmathsetmacro{\y}{\hiddencenter - (\i - 1)} % 根据隐藏层的中点调整y坐标
    % \pic[shift={(5,\y,5)}] at (5,\y,5) {Ball={name=hidden\i, fill=green, opacity=0.6, radius=2.5, logo=$h_\i$}};
    \pic[shift={(5,\y,5)}] at (0,\y,0) {Ball={name=hidden\i, fill=green, opacity=0.6, radius=2.5, logo=$h_\i$}};
}
\node[font=\small, align=center] at (5, -6, 5) {隐藏层}; % 居中调整字体大小

% 输出层节点位置
\foreach \i in {1,...,\numoutputs}
{
    \pgfmathsetmacro{\y}{\outputcenter - (\i - 1)} % 根据输出层的中点调整y坐标
    % \pic[shift={(10,\y,10)}] at (10,\y,10) {Ball={name=output\i, fill=red, opacity=0.6, radius=2.5, logo=$y_\i$}};
    \pic[shift={(10,\y,5)}] at (0,\y,0) {Ball={name=output\i, fill=red, opacity=0.6, radius=2.5, logo=$y_\i$}};
}
\node[font=\small, align=center] at (10, -2, 5) {输出层}; % 居中调整字体大小

% 连接输入层到隐藏层
\foreach \i in {1,...,\numinputs}
{
    \foreach \j in {1,...,\numhidden}
    {
        \draw [connection]  (input\i-east) -- node {\copymidarrow} (hidden\j-west);
    }
}

% 连接隐藏层到输出层
\foreach \i in {1,...,\numhidden}
{
    \foreach \j in {1,...,\numoutputs}
    {
        \draw [connection]  (hidden\i-east) -- node {\copymidarrow} (output\j-west);
    }
}

%%%%%%%%%%%%%%%%%%%%%%%%%%%%%%%%%%%%%%%%%%%%%%%%%%%%%%%%%%%%%%%%%%%%%%%%%%%%%%%%%%%%%%%
\end{tikzpicture}
\end{document}
