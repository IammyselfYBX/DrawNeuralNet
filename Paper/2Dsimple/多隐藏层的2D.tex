\documentclass[border=15pt, multi, tikz]{standalone}
\usepackage{ctex}  % 引入ctex包以支持中文
\usepackage{import}
\subimport{../../layers/}{init}  % 确保路径正确
\usetikzlibrary{positioning}

% 定义颜色
\definecolor{nodecolor}{rgb}{1, 0.7, 0.5}  % 浅黄色
\definecolor{linecolor}{rgb}{0.8, 0.1, 0.1}  % 深红色
\definecolor{somelinecolor}{rgb}{0.8, 0.2, 0.2}  % 特殊的红色调用于标记特定连线

\begin{document}
\begin{tikzpicture}[neuron/.style={circle, fill=nodecolor, minimum size=1em, text=white},
                    connect/.style={-latex, draw=linecolor},
                    specialconnect/.style={-latex, draw=somelinecolor, line width=1.5pt}]  % 定义特殊连线样式
    % 定义每层的节点数
    \def\inputnum{5} % 输入层节点数
    \def\hiddennum{6} % 隐藏层节点数
    \def\outputnum{2} % 输出层节点数

    % 绘制输入层节点
    \foreach \i in {1,...,\inputnum}
    {
        \node[neuron] (Input-\i) at (0,-2*\i em) {};
        \ifnum\i=2 \node at (0,-4em) {2}; \fi  % 在第二个节点上标记数字2
    }
    \node[font=\small, align=center] at (0, -12em) {输入层};

    % 绘制隐藏层h1节点
    \foreach \i in {1,...,\hiddennum}
    {
        \node[neuron] (Hidden1-\i) at (3,-2*\i em + 1em) {};
    }
    \node[font=\small, align=center] at (3, -13em) {隐藏层 h1};

    % 绘制隐藏层h2节点
    \foreach \i in {1,...,\hiddennum}
    {
        \node[neuron] (Hidden2-\i) at (6,-2*\i em + 1em) {};
    }
    \node[font=\small, align=center] at (6, -13em) {隐藏层 h2};

    % 绘制输出层节点
    \foreach \i in {1,...,\outputnum}
    {
        \node[neuron] (Output-\i) at (9,-5*\i em + 2em) {};
    }
    \node[font=\small, align=center] at (9, -12em) {输出层};

    % 连接输入层和隐藏层h1
    \foreach \i in {1,...,\inputnum}
    {
        \foreach \j in {1,...,\hiddennum}
        {
            \ifnum\i=2 \ifnum\j=3
                \draw[specialconnect] (Input-\i) -- (Hidden1-\j);  % 使用特殊颜色标记连接
            \else
                \draw[connect] (Input-\i) -- (Hidden1-\j);
            \fi
            \else
                \draw[connect] (Input-\i) -- (Hidden1-\j);
            \fi
        }
    }

    % 连接隐藏层h1和h2
    \foreach \i in {1,...,\hiddennum}
    {
        \draw[connect] (Hidden1-\i) -- (Hidden2-\i);  % 一一对应连接
    }

    % 连接隐藏层h2和输出层
    \foreach \i in {1,...,\hiddennum}
    {
        \foreach \j in {1,...,\outputnum}
        {
            \draw[connect] (Hidden2-\i) -- (Output-\j);
        }
    }
\end{tikzpicture}
\end{document}
