\documentclass[border=15pt, multi, tikz]{standalone}
\usepackage{ctex}  % 引入ctex包以支持中文
\usepackage{import}
\subimport{../../layers/}{init}  % 确保路径正确
\usetikzlibrary{positioning}

% 定义颜色
\definecolor{nodecolor}{rgb}{1, 0.7, 0.5}  % 浅黄色
\definecolor{linecolor}{rgb}{0.8, 0.1, 0.1}  % 深红色
\definecolor{somelinecolor}{rgb}{0.8, 0.2, 0.2}  % 不同的红色调

\begin{document}
\begin{tikzpicture}[neuron/.style={circle,fill=nodecolor,minimum size=1em,text=white},
                    connect/.style={-latex, draw=linecolor}] % 直接使用定义的颜色名
    % 定义每层的节点数
    \def\inputnum{5} % 输入层节点数
    \def\hiddennum{6} % 隐藏层节点数
    \def\outputnum{2} % 输出层节点数

    % 绘制输入层节点
    \foreach \i in {1,...,\inputnum}
    {
        \node[neuron] (Input-\i) at (0,-2*\i em) {}; % 控制间距为2em
    }
    \node[font=\small, align=center] at (0, -12em) {输入层}; % 居中调整字体大小

    % 绘制隐藏层节点
    \foreach \i in {1,...,\hiddennum}
    {
        \node[neuron] (Hidden-\i) at (3,-2*\i em + 1em) {};
    }
    \node[font=\small, align=center] at (3, -13em) {隐藏层}; % 居中调整字体大小

    % 绘制输出层节点
    \foreach \i in {1,...,\outputnum}
    {
        \node[neuron] (Output-\i) at (6,-5*\i em + 2em) {};
    }
    \node[font=\small, align=center] at (6, -12em) {输出层}; % 居中调整字体大小

    % 连接输入层和隐藏层
    \foreach \i in {1,...,\inputnum}
    {
        \foreach \j in {1,...,\hiddennum}
        {
            \draw[connect] (Input-\i) -- (Hidden-\j);
        }
    }

    % 连接隐藏层和输出层
    \foreach \i in {1,...,\hiddennum}
    {
        \foreach \j in {1,...,\outputnum}
        {
            \draw[connect] (Hidden-\i) -- (Output-\j);
        }
    }
\end{tikzpicture}
\end{document}
